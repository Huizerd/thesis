\documentclass{propositions}

%% Turn off page numbering for the propositions and make the margins on both
%% sides equal and symmetrical.
\geometry{twoside=false}
\pagestyle{empty}

% \usepackage{fontspec}
% \defaultfontfeatures{Path = /usr/local/texlive/2024/texmf-dist/fonts/opentype/public/fontawesome/}
% \usepackage{fontawesome}

% \setmainfont[Path = fonts/libertinus/, ItalicFont=libertinusserif-italic.otf, BoldFont=libertinusserif-bold.otf, BoldItalicFont=libertinusserif-bolditalic.otf]{libertinusserif-regular.otf}
% \setsansfont[Path = fonts/libertinus/, BoldFont=libertinussans-bold.otf, ItalicFont=libertinussans-italic.otf]{libertinussans-regular.otf}
% \setmathfont[Path = fonts/libertinus/]{libertinusmath-regular.otf}
% \setmonofont[Scale=MatchLowercase, Path=fonts/inconsolata/]{Inconsolata-Regular.ttf}
%% The default style for text is Tahoma (sans-serif).
%\renewcommand*\familydefault{\rmfamily}

\begin{document}

%% Specify the title and author of the thesis. This information will be used on
%% both the English and Dutch side and in the metadata of the final PDF.
\title{Self-Supervised Neuromorphic Perception\\\vspace{1pt} for Autonomous Flying Robots}
\author{Federico}{Paredes-Vallés}

\begin{center}

{\Large\titlefont\bfseries Propositions}

\medskip

accompanying the dissertation

\medskip

%% Print the title.
{\makeatletter
\titlestyle\bfseries\large\@title
\makeatother}

%% Print the optional subtitle.
{\makeatletter
\ifx\@subtitle\undefined\else
    \titlefont\titleshape\@subtitle
\fi
\makeatother}

\smallskip

by

\smallskip

%% Print the full name of the author.
\makeatletter
{\large\titlefont\bfseries\@firstname\ {\titleshape\@lastname}}
\makeatother

\end{center}

\vspace{2.5pt}

\begin{enumerate}
\itemsep0em 
\item Embracing simplicity and interpretability as core tenets in engineering nurtures effective solutions, mitigating complexity risks, and cultivating a profound system understanding. [This thesis]
%\item Self-supervised learning will drive a transformative shift in the future of artificial intelligence, enabling machines to autonomously learn and adapt in complex real-world environments. [This thesis]
\item Adopting self-supervised learning in robotics harnesses the power of real, unlabeled data, bypassing the need for sensor simulation and hence addressing the reality gap in perception systems. [This thesis]
\item Neuromorphic technology has the potential to catalyze intelligence democratization in robotics, revolutionizing small robot integration, empowering advanced capabilities, and expanding their applications. [This thesis]
\item Nature, as source of inspiration for robotics, guides a path towards practical solutions when embraced as a muse rather than a blueprint. [This thesis]
\item Event cameras, driven by the essence of change, unveil a profound truth in robotics: perception thrives on dynamism. 
\item In an AI-dominated media landscape, education, logic, and critical thinking will remain as our last refuge, protecting truth and upholding integrity.
\item The journey to become an independent researcher transcends solo pursuits, as fruitful collaborations cultivate breakthrough innovation and amplify the impact of our work.
\item In a multi-cultural work environment, excelling in one's job and embracing cross-cultural understanding are two sides of the same coin.
% Guido: not entirely clear to me: do you need to excell in your job in order to understand people from another culture?
%\item In the realm of creativity, where pixels and ideas collide, videogames and academic research emerge as parallel universes, inviting exploration, discovery, and the unleashing of untapped potentials. 
\item In the realm of creativity, where pixels and ideas collide, videogames and computer vision research coexist as distinct yet interconnected domains, both inviting exploration, discovery, and the unleashing of untapped potentials.
\item Preserving personal time poses a formidable challenge for Ph.D.\ candidates, transcending research domains while being integral to academic endeavors.
% Guido: what do you mean with this challenge transcending research domains? That it is very hard?
\end{enumerate}

% Christophe: I would however like to ask if you can still explore if a more specific phrasing of 2 and 3 would be possible (SSL and neuromorph.) in the context of being measurable, defendable and opposable.  Will drive a shift = prediction = difficult to defend/oppose: e.g. politics can make a law to prohibit it and then it will not happen. Can you be more specific and for instance state which technical aspect (measurable) is better (defendable) than the state-of-the-art in which way.

% Christophe: The phrasing of proposition 9 is almost a poem. Beautiful, but also complex and risky. It think you mean: creativity with pixels has two parts: games and academic research: both contain exploration, discovery and benefit com unleashing of potentials. I have a few problems with the proposition: first and foremost, you are literally saying that academic research emerges as a parallel universe (you probably mean parallel to games, but it can be read as parallel to the universe). While this is very defendable for video games (which are imaginary digital worlds), the point of academic research is to unveil reality and propose solutions to its challenges, so calling it a parallel universe is at least a bit confusing. Also, the "where pixels and ideas collide" is a beautiful metaphor, but applies more to video games than "academic research in general". At least then narrow it down to computer vision research for instance. Finally, as phrased it looks like games and research are the only two parts of creativity with pixels, which sounds incomplete to me. I do agree that the exploration, discovery, unleashing of potentials apply to both parts. Or maybe you mean with the word universe that the environment is quite detached or different in mindset than most of society, and then I would agree, but I would like to ask to avoid the misunderstanding that academic research lives in a different universe.

% Christophe: 8. It is only possible to excel at one's job in a multi-cultural environment if you also excel in multi-cultural understanding.
% Christophe: 10. Personal time is simultaneously important for nurturing ideas and hard to arrange during a Ph.D.

\vspace{2.5pt}

%% Apart from the name and title of the supervisor, the following text is
%% dictated by the promotieregelement.
\begin{center}
\noindent These propositions are regarded as opposable and defendable, and have been approved as such by the
promotors prof.\ dr.\ G.\ C.\ H.\ E.\ de Croon, and \\dr.\ ir.\ C.\ De Wagter.
\end{center}

%% \clearpage
%% {\selectlanguage{dutch}

%% \begin{center}

%% {\Large\titlefont\bfseries Stellingen}

%% \bigskip

%% behorende bij het proefschrift

%% \bigskip

%% %% Print the title.
%% {\makeatletter
%% \titlestyle\bfseries\large\@title
%% \makeatother}

%% %% Print the optional subtitle.
%% {\makeatletter
%% \ifx\@subtitle\undefined\else
%%     \titlefont\titleshape\@subtitle
%% \fi
%% \makeatother}

%% \bigskip

%% door

%% \bigskip

%% %% Print the full name of the author.
%% \makeatletter
%% {\large\titlefont\bfseries\@firstname\ {\titleshape\@lastname}}
%% \makeatother

%% \end{center}

%% \bigskip
%% \bigskip

%% \begin{enumerate}

%% \item Stelling 1.
%% \item Stelling 2.
%% \item Stelling 3.
%% \item Stelling 4.
%% \item Stelling 5.
%% \item Stelling 6.
%% \item Stelling 7.
%% \item Stelling 8.
%% \item Stelling 9.
%% \item Stelling 10.

%% \end{enumerate}

%% \bigskip
%% \bigskip

%% %% Apart from the name and title of the supervisor, the following text is
%% %% dictated by the promotieregelement.
%% \begin{center}
%% Deze stellingen worden opponeerbaar en verdedigbaar geacht en zijn als zodanig goedgekeurd door de promotoren prof.\ dr.\ A.\ van Deursen and dr.\ A.\ Zaidman.
%% \end{center}

%% }

\end{document}

